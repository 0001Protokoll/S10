%\input{usepackage.tex}

 
%\begin{document}
\section{Zusammenfassung}
Durch die Aufnahme des Spektrums von einer rauschenden Butan-Flamme wurden die Übergänge des $C_2-Radikals$ beobachtet. Die Charakteristischen  Konstanten: $\nu_e$, die harmonischen Schwingungskonstanten und die Anharmonizitätskonstanten des Grund,sowie des angeregten Zustandes wurden anschießend mit einer vorimplementierten Routine durch Vorgabe einzelner Banden errechnet.


\begin{table}[H]

 
 \caption{Zusammenfassung der Ergebnisse des Fits zur Bestimmung der Konstanten. Alle Werte sind in $\si{cm}^{-1}$ angegeben.}
\begin{tabular}{C{0.3\linewidth}|C{0.3\linewidth}C{0.3\linewidth}}

 
 Konstante &  Messwert &  Literatur $^{[1]}$\\
  \hline \addlinespace[1ex] 
$\nu_e$ & $18813 \pm 5.45$ & \\
$\omega'_e$ & $1544.7 \pm9.92$ & $1641.35$ \\
$\omega'_e x'_e$ & $25.5 \pm 3.0$ &  $11.67$ \\
$\omega''_e$ & $1649.2 \pm 9.92$ & $1788.22$ \\
$\omega''_e x''_e$ & $28.75 \pm 4.18$ & $16.440$ \\
 
   
 \end{tabular}
 \label{tab1}
 \end{table}

Durch die erhaltenen Schwingungskonstanten wurde gemäß Formen \ref{1} bestimmt und in Tabelle \ref{tab2} aufgetragen.Die vorliegende Abweichung zu den Literarurwerten lässt sich durch die empirische Wahl der Extrema in Abbildung \ref{Bunsen} begründen. Diese Extrema bzw. markante Banden wurden zur Bestimmung der Schwingungskonstanten herangezogen. Dieser Fehler kann vermieden werden, indem die \textit{Datenwave} durch einen geeigneten Algorithmus in bestimmten Wellenzahl Bereich auf Extrema analysiert wird. Insbesondere werden Fehler bei der Kalibrierung durch den gesamten Versuch fortgepflanzt, also auch eine zusätliche Unsicherheit in der Wahl der Extrema, basierend auf einer falschen Skalierung, bewirken.

%\end{document}


