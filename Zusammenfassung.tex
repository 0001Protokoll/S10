%%\documentclass[a4paper, 12pt]{scrreprt}

\documentclass[a4paper, 12pt]{scrartcl}
%usepackage[german]{babel}
\usepackage{microtype}
%\usepackage{amsmath}
%usepackage{color}
\usepackage[utf8]{inputenc}
\usepackage[T1]{fontenc}
\usepackage{wrapfig}
\usepackage{lipsum}% Dummy-Text
\usepackage{multicol}
\usepackage{alltt}
%%%%%%%%%%%%bis hierhin alle nötigen userpackage
\usepackage{tabularx}
\usepackage[utf8]{inputenc}
\usepackage{amsmath}
\usepackage{amsfonts}
\usepackage{amssymb}

%\usepackage{wrapfig}
\usepackage[ngerman]{babel}
\usepackage[left=25mm,top=25mm,right=25mm,bottom=25mm]{geometry}
%\usepackage{floatrow}
\setlength{\parindent}{0em}
\usepackage[font=footnotesize,labelfont=bf]{caption}
\numberwithin{figure}{section}
\numberwithin{table}{section}
\usepackage{subcaption}
\usepackage{float}
\usepackage{url}
%\usepackage{fancyhdr}
\usepackage{array}
\usepackage{geometry}
%\usepackage[nottoc,numbib]{tocbibind}
\usepackage[pdfpagelabels=true]{hyperref}
\usepackage[font=footnotesize,labelfont=bf]{caption}
\usepackage[T1]{fontenc}
\usepackage {palatino}
%\usepackage[numbers,super]{natbib}
%\usepackage{textcomp}
\usepackage[version=4]{mhchem}
\usepackage{subcaption}
\captionsetup{format=plain}
\usepackage[nomessages]{fp}
\usepackage{siunitx}
\sisetup{exponent-product = \cdot, output-product = \cdot}
\usepackage{hyperref}
\usepackage{longtable}
\newcolumntype{L}[1]{>{\raggedright\arraybackslash}p{#1}} % linksbündig mit Breitenangabe
\newcolumntype{C}[1]{>{\centering\arraybackslash}p{#1}} % zentriert mit Breitenangabe
\newcolumntype{R}[1]{>{\raggedleft\arraybackslash}p{#1}} % rechtsbündig mit Breitenangabe
\usepackage{booktabs}
\renewcommand*{\doublerulesep}{1ex}
\usepackage{graphicx}




 
%\begin{document}
\section{Zusammenfassung}


Es wurden die elektronischen Übergängen von dem $C_2-Radikal$, welches bei der Verbrennung von Butan entsteht, untersucht. Hierfür wurde das Gitter mit Hilfe einer Natriumdampflampe kalibriert, indem die Gitterpositionsspannung bei der nullten und ersten Ordnung aufgenommen wurde. Aus dessen Differenz und der Wellenlänge der Natrium-D-Linie konnte 
\begin{equation}
\frac{dU}{d\lambda}=10.03~\si{\frac{mV}{nm}}
\end{equation}
bestimmt werden. Durch die Aufnahme des Spektrums von einer rauschenden Butan-Flamme wurden die Übergänge des $C_2-Radikals$ beobachtet. Die Charakteristischen  Konstanten: $\nu_e$, die harmonischen Schwingungskonstanten und die Anharmonizitätskonstanten des Grund,sowie des angeregten Zustandes wurden anschießend mit einer vorimplementierten Routine durch Vorgabe einzelner Banden errechnet.


\begin{table}[H]

 
 \caption{Zusammenfassung der Ergebnisse des Fits zur Bestimmung der Konstanten. Alle Werte sind in $\si{cm}^{-1}$ angegeben.}
\begin{tabular}{C{0.3\linewidth}|C{0.3\linewidth}C{0.3\linewidth}}

 
 Konstante &  Messwert &  Literatur $^{[1]}$\\
  \hline \addlinespace[1ex] 
$\nu_e$ & $18813 \pm 5.45$ & \\
$\omega'_e$ & $1544.7 \pm9.92$ & $1641.35$ \\
$\omega'_e x'_e$ & $25.5 \pm 3.0$ &  $11.67$ \\
$\omega''_e$ & $1649.2 \pm 9.92$ & $1788.22$ \\
$\omega''_e x''_e$ & $28.75 \pm 4.18$ & $16.440$ \\
 
   
 \end{tabular}
 \label{tab1}
 \end{table}

Hierdurch war es nun möglich alle weiteren Übergänge zu berechnen. Diese sind in der Deslandres-Tabelle zusammengefasst:

\begin{table}[H]
 
\caption{Deslandres-Tabelle der beobachteten Übergänge in $\si{cm}^{-1}$. Die
Werte gehen aus der Berechnung mit Gleichung \ref{1} hervor}
\begin{tabular}{L{0.1\linewidth}|L{0.15\linewidth}L{0.15\linewidth}L{0.15\linewidth}L{0.15\linewidth}L{0.15\linewidth}}

 
v''/v' & 0 & 1 & 2 & 3 & 4 \\
\hline \addlinespace[1ex]
$0$ & $18860 \pm 20$ & $20460 \pm 35$ & $21990 \pm60$ &  &  \\
$1$ & $17370 \pm 35$ & $18960 \pm 50$ & $20500\pm 80$ & $22000\pm 100$ &  \\
$2$ & $15930 \pm 55$ & $17520 \pm 70$ & $19100 \pm 100$ &  & \\
$3$ &  & $16150\pm100$ & $17700\pm 150$ &  & \\
$4$ &  &  & $16400\pm150$ & $17800\pm200$ & \\
$5$ &  &  &  & $16500\pm 250$ & \\ 
$6$ &  &  &  & & $16700\pm 350$\\  
 \end{tabular}
 \label{tab2}
 \end{table}



%\end{document}


