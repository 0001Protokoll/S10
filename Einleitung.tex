%\input{usepackage.tex}
%\begin{document}

\section{Einleitung}
Die Pyrolyse als allgemeine thermische Zersetzung einer chemischen Verbindung findet unbewusst weite Anwendung im Alltag. Jeder Gasherd sowie ebenfalls der Grill für ein BBQ basieren auf der Pyrolyse von Brenngas bzw. organischen Material (wie Holz), mit dem Ziel der punktuellen Termperaturerhöhung. Zu erwähnen ist auch, dass die menschliche Entwicklung (so vermuten Anthropologen) durch das nutzbar machen von geplanter Verbrennung und somit der Verfügbarkeit von großen Enerigemengen begünstigt wurde. Dies motiviert zu verstehen, was genau den Prozess der Pyrolyse möglich macht. Eine Art sich mit der Fragestellung zu befassen ist durch die spektroskopische Untersuchung des Verbrennungsprozesses, z.B bezogen auf den sichtbaren Spektralbereich. Ferner kann das Verständnis solcher Prozesse neue Erkentnisse über die Natur von kurzlebigen, reaktiven, hochtemperatur Spezies geben.  
%\end{document}
